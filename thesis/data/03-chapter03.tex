\chapter{\label{chap:03}The Derivation of Lorenz System}

The equations of motion are used in the cases come from the following three principles: the incompressibility, the momentum law and the Juor diffusion law.\\
\textbf{The incompressibility}
\begin{equation}
\frac{\partial u}{\partial x}+\frac{\partial v}{\partial y}+\frac{\partial w}{\partial z}=0.
\label{equ:incompressibility}
\end{equation}
\textbf{The momentum law (Naiver-Stokes equations):}
\begin{equation}
\rho\frac{\partial \boldsymbol{u}}{\partial t}+\rho\left(\boldsymbol{u}\cdot\boldsymbol{\nabla}\right)\boldsymbol{u}=-\boldsymbol{\nabla}p+\mu\nabla^2\boldsymbol{u}+\rho\boldsymbol{g}
\label{equ:naiver_stokes}
\end{equation}
\noindent \textbf{Juor diffusion law:}
\begin{equation}
\frac{\partial T}{\partial t}+\left(\boldsymbol{u}\cdot\boldsymbol{\nabla}\right)T=\kappa\nabla^2 T.
\label{equ:jour_diffusion}
\end{equation}
For two dimensional ``rolls'' in the \(x\)-\(z\) plane, the \(v\)-component vanishes everywhere. The equations become
\begin{align}
	\frac{\partial u}{\partial x}+\frac{\partial w}{\partial z}&=0.\label{equ:incompressible_roll}\\
	\rho\frac{\partial u}{\partial t}+\rho u\frac{\partial u}{\partial x}+\rho w\frac{\partial u}{\partial z}&=-\frac{\partial p}{\partial x}+\mu\nabla^2 u.\label{equ:navier_stokes_roll_1}\\
	\rho\frac{\partial w}{\partial t}+\rho u\frac{\partial w}{\partial x}+\rho w\frac{\partial w}{\partial z}&=
		-\frac{\partial p}{\partial z}+\mu\nabla^2 w-g\rho.\label{equ:navier_stokes_roll_2}\\
	\frac{\partial T}{\partial t}+u\frac{\partial T}{\partial x}+w\frac{\partial T}{\partial z}&=\kappa\nabla^2 T.\label{equ:heat_defusion_roll}
\end{align}
Equation~\ref{equ:incompressible_roll} can be satisfied if \(\psi\) is represented by a stream function 
\begin{equation}
	u=-\frac{\partial \psi}{\partial z},\quad w=\frac{\partial \psi}{\partial x}.
	\label{equ:stream_function}
\end{equation}
As for the temperature field, we split it into the static temperature profile and a perturbation \(\theta\left(x,z,t\right)\), i.e.
\begin{equation}
	T\left( t,x,z \right)=\left( T_0-\beta z \right)+\theta\left(x,z,t\right).
	\label{equ:temperature_expansion}
\end{equation}
Here, \(T_0\) is the temperature of middle plane. \(\beta\) is the gradient of the static linear temperature profile, defined as
\begin{equation}
	\beta=\frac{\Delta T_0}{H}.
	\label{equ:temperature_gradient}
\end{equation}
The pressure \(p\) can be eliminated by using the \(z\)-derivative of Equation~\ref{equ:navier_stokes_roll_1} to subtract the \(x\)-derivative of Equation~\ref{equ:navier_stokes_roll_2}. And by substituting the stream function, we obtain
\begin{align}
	\frac{\partial}{\partial t}\nabla^2\psi-\left(\frac{\partial}{\partial z}\psi\right)\left(\frac{\partial}{\partial x}\nabla^2\psi\right)+\left(\frac{\partial}{\partial x}\psi\right)\left(\frac{\partial}{\partial z}\nabla^2\psi\right)-g\alpha\frac{\partial\theta}{\partial x}-\nu\nabla^4\psi&=0.\label{equ:stream_equation}\\
	\frac{\partial}{\partial t}\theta-\left(\frac{\partial}{\partial z}\psi\right)\left( \frac{\partial}{\partial x}\theta \right)+\left(\frac{\partial}{\partial x}\psi\right)\left( \frac{\partial}{\partial z}\theta \right)-\beta\frac{\partial}{\partial x}\psi-\kappa\nabla^2\theta&=0.\label{equ:thermal_equation}
\end{align}
This is Saltzmann's nonlinear equations. The foregoing equations can be translated to dimensionless form
\begin{align}
	\frac{\partial}{\partial t}\nabla^2\psi-\left(\frac{\partial}{\partial z}\psi\right)\left(\frac{\partial}{\partial x}\nabla^2\psi\right)+\left(\frac{\partial}{\partial x}\psi\right)\left(\frac{\partial}{\partial z}\nabla^2\psi\right)-\sigma\frac{\partial\theta}{\partial x}-\sigma\nabla^4\psi&=0.\label{equ:stream_equation_dimensionless}\\
	\frac{\partial}{\partial t}\theta-\left(\frac{\partial}{\partial z}\psi\right)\left( \frac{\partial}{\partial x}\theta \right)+\left(\frac{\partial}{\partial x}\psi\right)\left( \frac{\partial}{\partial z}\theta \right)-R\frac{\partial}{\partial x}\psi-\nabla^2\theta&=0.\label{equ:thermal_equation_dimensionless}
\end{align}
under the transformation
\begin{equation}
	\left\{
	\begin{matrix}
	x\rightarrow Hx\\
	z\rightarrow Hx\\
	t\rightarrow\left( H^2/\kappa \right)t\\
	\psi\rightarrow \kappa\psi\\
	\theta\rightarrow \left( \kappa\nu/g\alpha H^3 \right)\theta\\
	\end{matrix}
	\right. .
\end{equation}
Here, the Prandtl number and the Rayleigh number first appear
\begin{equation*}
	\left\{\begin{matrix}\sigma=\frac{\nu}{\kappa}\\
	R=\frac{g\alpha\beta H^4}{\kappa\nu}\end{matrix}\right. .
\end{equation*}
The result for case (a) is that for a critical minimum value of Rayleigh number,
\begin{equation*}
	R=R_c=\frac{27}{4}\pi^4.
\end{equation*}
a steady solution of the form
\begin{align*}
	\psi&=\frac{1}{a}A\sin\left(ax\right)\sin\left(\pi z\right);\\
	\theta&=\frac{\left(\pi^2+a^2\right)^2}{a^2}B\cos\left(ax\right)\sin\left(\pi z\right).
\end{align*}
can satisfy the boundary conditions. If we substitute them to Equation~\ref{equ:stream_equation_dimensionless}-\ref{equ:thermal_equation_dimensionless}, one of the term will throw out
\begin{equation*}
	-\left(\frac{\partial}{\partial z}\psi\right)\left( \frac{\partial}{\partial x}\theta \right)+\left(\frac{\partial}{\partial x}\psi\right)\left( \frac{\partial}{\partial z}\theta \right)=\frac{\left(\pi^2+a^2\right)^2\pi}{2a^2}AB\sin\left( 2\pi z \right)
\end{equation*}
whose wave number is other than the critical mode. The appearance of this term suggests it should also be put into the solution to get more exact approach
\begin{align*}
	\psi&=\frac{1}{a}A\sin\left(ax\right)\sin\left(\pi z\right);\\
	\theta&=\frac{\left(\pi^2+a^2\right)^2}{a^2}B\cos\left(ax\right)\sin\left(\pi z\right)-\frac{\left(\pi^2+a^2\right)^2\pi}{2a^2}C\sin\left( 2\pi z \right).
\end{align*}
So that the equations will be transformed to
\begin{align*}
	-\frac{\pi^2+a^2}{a}\frac{\mathrm{d}A}{\mathrm{d}t}+\frac{\left(\pi^2+a^2\right)^2}{a}\sigma B-\frac{\left(\pi^2+a^2\right)^2}{a}\sigma A&=0.\\
	\frac{\left(\pi^2+a^2\right)^2}{a^2}\frac{\mathrm{d}B}{\mathrm{d}t}+\frac{\left(\pi^2+a^2\right)^2\pi^2}{2a^2}AC-RA+\frac{\left(\pi^2+a^2\right)^3}{a^2}B&=0.\\
	-\frac{\left(\pi^2+a^2\right)^2\pi}{2a^2}\frac{\mathrm{d}C}{\mathrm{d}t}+\frac{\left(\pi^2+a^2\right)^2\pi}{2a^2}AB-\frac{4\left(\pi^2+a^2\right)^2\pi^3}{2a^2}C&=0.
\end{align*}
By rescale them with
\begin{equation}
	\left\{
		\begin{matrix}
			t=\left(\frac{3\pi^2}{2}\right)^{-1}\tau\\
			A=3X\\
			B=\frac{27\pi^3}{2\sqrt{2}}Y\\
			C=\frac{27\pi^3}{4}Z\\
		\end{matrix}
	\right.
	\label{equ:rescale}
\end{equation}
In the end, the Lorenz system can be obtained
\begin{equation}
	\frac{\mathrm{d}}{\mathrm{d}\tau}
	\begin{pmatrix}
		X\\Y\\Z
	\end{pmatrix}
	=
	\begin{pmatrix}
	&&-&\sigma X&+&\sigma Y&&\\-&XZ&+&rX&-&Y&&\\&XY&&&&&-&bZ\\
	\end{pmatrix}
	\label{equ:lorenz_system}
\end{equation}
\par
